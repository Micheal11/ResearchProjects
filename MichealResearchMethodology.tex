\documentclass{article}
\begin{document}

\begin {small} \center \textbf{OMARA MICHEAL    215001305     15/U/1102 }\end{small}


\begin{Large}\textbf{SOME PASSENGERS ESCAPE PAYING TRANSPORT FARE ON BOARDING TAXIS.}\end{Large}
\section{Abstract.}
{The main objective of this report is to alter out the skills a sharp person can use to resist from paying transport fare if he/she happens to board a taxi traveling short journeys. This may be used   in case the passenger does not have enough funds to cater for his/her basic necessities of life. Therefore, to favor his/her status, there is need to endeavor to tricks that are explained in the report.}
\section{Introduction.}
Traveling is one of the most regular activities that happen in each of the districts in the country. Majority of the citizens of Uganda travel using public transport means because they can only afford fares charged by those means other than opting for buying personal vehicles which are expensive to service regularly. One of the public transport means used is a taxi which can accommodate fourteen passengers at time as its legal license confirms.

Research highlights it clearly that majority of the taxi passengers are the youth of the country. This is because most of the youth can persevere whichever situation they come across to satisfy their utility.

However , there is a high level of unemployment prevailing in the country and majority of the youth suffer the Baden because they lack job opportunities making their cost of living high. Therefore because of this, majority resort to involving in bad offenses because nothing they can do to improve their standards of living. 


\section{Methodology.}
{This research was conducted by interacting with taxi drivers on different stages in the suburbs of the city as well as asking taxi conductors plus a few conductresses from which I acquired this information concerning the challenge they face while transporting passengers. }

\subsection{HOW THEY EVADE PAYING TRANSPORT FARE?}
{On boarding a taxi, some passengers keep up changing seats to make sure that the conductors do not trace their initial positions as they had just entered. They keep up trying helping in forwarding other passengers fare who are seated at the extreme back of the taxi where the conductor can reach easily if he requests the passengers to pay money. On reaching the destination, the wise guy moves out of the taxi gently without  paying the fare and if the conductor asks him to pay, he replies that “I am the person who gave you money when we were almost amidst of the journey”.}

\subsection{HOW THEY MAKE THEIR TRICK PERFECT?}
{They make sure that they are neither among the first nor the last passengers to move out of the taxi which makes it difficult for the conductors to trace their faces on whether they had already paid or not.
They also keep up changing positions while in the taxi in order to confuse the conductors` attention . This gives them a high advantage since they are ever showing  confidence in every step they make next.
}

\section{Conclusion}
{Passengers using this trick should stop it as many conductors are now alert of the trick being used since it makes them incur losses as well as leaving them with a negative attitude. }




\end{document}